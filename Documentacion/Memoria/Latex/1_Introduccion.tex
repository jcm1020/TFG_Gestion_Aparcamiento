\documentclass[12pt,a4paper]{article}
\usepackage[utf8]{inputenc}
\usepackage[spanish]{babel}
\usepackage{enumitem}
\usepackage{setspace}
\usepackage{lmodern}
\usepackage{hyperref}

\setlist[itemize]{noitemsep, topsep=0pt}
\setstretch{1.15}

% Comando para simular "Capítulo"
\newcommand{\capitulo}[2]{%
	\section*{Capítulo #1. #2}%
}

\begin{document}
	
	\capitulo{1}{Introducción}
	
	Descripción del contenido del trabajo y de la estructura de la memoria y del resto de materiales entregados.
	
	\section{Objetivo principal del sistema}
	
	Diseñar e implementar una aplicación que, a partir de imágenes capturadas por una o varias cámaras instaladas en un aparcamiento, detecte y estime en tiempo real el número de plazas ocupadas y libres, mostrando la información y localización en un panel web y/o app móvil. El sistema deberá contemplar condiciones reales (variaciones de luz, oclusiones, diferentes ángulos y resoluciones), y asegurar escalabilidad (varias cámaras) y robustez (detección estable), así como privacidad (no identificación de personas ni matrículas).
	
\end{document}

