\documentclass[12pt,a4paper]{article}
\usepackage[utf8]{inputenc}
\usepackage[spanish]{babel}
\usepackage{enumitem}
\usepackage{setspace}
\usepackage{lmodern}
\usepackage{hyperref}

\setlist[itemize]{noitemsep, topsep=0pt}
\setstretch{1.15}

% Comando opcional para simular "Capítulo"
\newcommand{\capitulo}[2]{%
	\section*{Capítulo #1. #2}%
}

\begin{document}
	
	\capitulo{2}{Objetivos del proyecto}
	
	Este apartado explica de forma precisa y concisa cuáles son los objetivos que se persiguen con la realización del proyecto. 
	Se puede distinguir entre los objetivos marcados por los requisitos del software a construir y los objetivos de carácter técnico 
	que plantea a la hora de llevar a la práctica el proyecto.
	
	\section{Objetivo general}
	
	El objetivo de este sistema es crear el inicio de un sistema ampliable futuro que actúe como base para la gestión de espacios físicos. Tratara de abstraer la complejidad de percibir, analizar sobre un entorno del mundo real, permitiendo a cualquier organización “programar” sus operaciones “físicas” con facilidad.
	
	\section{Problema planteado}
	
	La gestión de espacios físicos complejos en agricultura, logística, aviación, parkings, etc. depende de sistemas específicos 
	para cada sector que se tratan de forma independiente. La falta de una plataforma \textbf{unificada} obliga a desarrollar soluciones desde cero, ralentizando la innovación y limitando la eficiencia en la puesta en marcha de nuevos proyectos.
	
	\section{Solución planteada: un sistema integral}
	
	Este sistema es una solución modular basada en tres componentes fundamentales que trabajan en conjunto:
	
	\begin{itemize}
		\item \textbf{El Ojo (Núcleo, Bloque visor):} Es el sistema de percepción. 
		Se conecta principalmente a cámaras IP para “ver” el espacio, identificando objetos y zonas. Su función es traducir el mundo físico a un \textbf{flujo de datos estructurado} y en tiempo casi real. 
		Los \textbf{\textit{DATOS}}.
		
		\item \textbf{Las reglas (Bloque lógico):} Es el procesador del sistema. Permite definir reglas de negocio tipo “SI esto ocurre, ENTONCES pasa esto otro”. Procesa los datos del Ojo para entender el contexto y generar eventos o alertas basados en la lógica definida por el usuario. La \textbf{\textit{INFORMACIÓN}}.
		
		\item \textbf{Las acciones, la aplicación en sí (Bloque ejecutivo):} Es el sistema de ejecución. 
		Conecta con el mundo real a través de integraciones (APIs) que permiten realizar acciones concretas: 
		enviar notificaciones, activar alarmas, llamar a APIs externas o controlar dispositivos físicos. 
		El \textbf{\textit{CONOCIMIENTO}}.
	\end{itemize}
	
	\section{Mercado y Aplicaciones}
	
	El sistema integral es una solución horizontal aplicable a múltiples sectores. 
	Los nichos iniciales se centran en entornos de alto valor y complejidad operativa:
	
	\begin{itemize}
		\item \textbf{Agricultura de Precisión:} Optimización de recursos, monitorización de cultivos y gestión de maquinaria.
		\item \textbf{Aeropuertos y Aeródromos:} Seguridad en pistas, gestión de aeronaves y optimización de operaciones.
		\item \textbf{Terminales Marítimas:} Control de contenedores, eficiencia en muelles y gestión de flujos de vehículos.
		\item \textbf{Almacenes y Fábricas:} Automatización logística, seguridad laboral y optimización de rutas.
		\item \textbf{Centros Comerciales:} Análisis de flujos de clientes, gestión de colas y optimización de la experiencia de compra.
	\end{itemize}
	
	\section{Modelo de Negocio}
	
	El modelo se basa en la provisión de un servicio, no en la venta de un producto:
	
	\begin{itemize}
		\item \textbf{SaaS:} Acceso a la plataforma bajo un modelo de pago por uso (\textit{pay-as-you-go}), 
		facturando según los recursos consumidos (cámaras, datos procesados, reglas ejecutadas).
		\item \textbf{Soluciones Sectoriales:} Paquetes de software específicos para cada nicho de mercado, 
		con reglas e integraciones predefinidas.
		\item \textbf{Marketplace:} Ecosistema donde terceros pueden crear y vender “Skills” o \textit{plugins} 
		para ampliar la funcionalidad de la plataforma, generando una nueva fuente de ingresos a través de comisiones.
	\end{itemize}
	
	\section{Ideas aplicadas a la gestión de parkings}
	
	\begin{itemize}
		\item \textbf{Gestión Inteligente de la Iluminación (un parking verde y eficiente): Sostenibilidad.}\\
		\textit{Cómo funcionaría:} Las cámaras detectan movimiento de personas o vehículos en una zona concreta 
		y activan la iluminación de esa zona al 100\%. El resto del parking se mantiene en un nivel bajo de luz, 
		ahorrando costes eléctricos.
		
		\item \textbf{Análisis de Comportamiento Anómalo.}\\
		\textit{Cómo funcionaría:} El sistema no solo mira coches, mira personas y movimientos. 
		Puede detectar a individuos que dan vueltas alrededor de coches sin una intención clara de entrar o salir, 
		midiendo el tiempo de estancia.
		
		\item \textbf{Agente de Tráfico Urbano.}\\
		\textit{Cómo funcionaría:} El sistema comparte datos en tiempo real con el centro de control de tráfico de la ciudad. 
		Se pueden enviar alertas como: “Atención, en los próximos 10 minutos se esperan 50 salidas simultáneas del parking X, 
		lo que provocará saturación en la calle Y”. El semáforo de esa salida puede ajustar sus tiempos de forma anticipada 
		para fluidificar el tráfico.
	\end{itemize}
	
	\section{Conclusión}
	
	El sistema no es una aplicación específica sin más, sino una capa de infraestructura tecnológica escalable 
	y aplicable a diversas soluciones. Crea un estándar para la gestión de espacios físicos con potencial de mercado, 
	automatizando y facilitando el control de operaciones del mundo real.
	
\end{document}

